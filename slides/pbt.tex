\documentclass{beamer}
\usepackage[utf8]{inputenc}
\usepackage[T1]{fontenc}
\usepackage{graphicx}
\usepackage{color}
\usepackage{hyperref}
\usepackage[all]{xy}

\setbeamertemplate{frametitle continuation}[from second]

% \mode<presentation>
% {
%   \usetheme{Warsaw}
%   % or ...

%   \setbeamercovered{transparent}
%   % or whatever (possibly just delete it)
% }


\title{Why property-based testing matters}

\author[Pedro Vasconcelos]{Pedro Vasconcelos \\ \texttt{pbvascon@fc.up.pt}}

\institute[LIACC, DCC/FCUP]{
  DCC/FCUP \& LIACC \\
  \includegraphics[width=0.4\textwidth]{images/fcup-identidade-logotipo-cores}
  \qquad
  \raisebox{4ex}{\includegraphics[width=0.3\textwidth]{images/liacc-logo.png}}
}


\newcommand{\bs}{\symbol{92}}

% discard trial for counter-example
\newcommand{\counter}[1]{\textbf{#1}}
\newcommand{\noncounter}[1]{{\mathtt{#1}}}

\newcommand{\conc}{\ensuremath{\!+\!\!+}}

% If you have a file called "university-logo-filename.xxx", where xxx
% is a graphic format that can be processed by latex or pdflatex,
% resp., then you can add a logo as follows:

% \pgfdeclareimage[height=0.5cm]{university-logo}{university-logo-filename}
% \logo{\pgfuseimage{university-logo}}



% Delete this, if you do not want the table of contents to pop up atr
% the beginning of each subsection:
%\AtBeginSubsection[]
%{
%  \begin{frame}<beamer>
%    \frametitle{Overview}
%    \tableofcontents[currentsection,currentsubsection]
%  \end{frame}
%}


% If you wish to uncover everything in a step-wise fashion, uncomment
% the following command: 

%\beamerdefaultoverlayspecification{<+->}


\begin{document}

\begin{frame}
  \titlepage
\end{frame}

\begin{frame}[allowframebreaks]
  \frametitle{Overview}
  
  \begin{itemize}
  \item A long-standing challenge for software engineering 
    is ensuring software correctness
  \item Formal proofs are (still) expensive and rarely used
  \item \emph{Tests} are the most commonly used
    practical technique 
  % \item \emph{Types} are another one (for another talk!)
  \item \emph{Unit tests} are the industry-standard
    for verification ``in the small''
  \end{itemize}

  \framebreak
  
  This talk:
  \begin{itemize}
  \item \emph{Property-based testing}: an
    automatic testing alternative to unit tests
  \item A ``lightweight'' formal method
  \item Available for many programming languages
  \item Many successful applications in open-source and industrial
    systems
  \item But still not commonly taught and under-utilized in practice
  \end{itemize}
  
  
\end{frame}

\begin{frame}[fragile]
  \frametitle{Unit tests}
\begin{itemize}
\item Code fragments for testing functions, classes, libraries, etc.
\item Express the expected behaviour for a specific combinations of inputs
\item Example: testing an integer square root function in Python
\begin{verbatim}
def test_isqrt():
   assert isqrt(0) == 0
   assert isqrt(2) == 1
   assert isqrt(4) == 2
   assert isqrt(5) == 2
   assert isqrt(9) == 3
\end{verbatim}
\end{itemize}


\end{frame}


\begin{frame}[fragile]
  \frametitle{Problems with unit tests}

  Cognitive bias:
  \begin{itemize}
  \item how can we include an edge case in the tests
    that we didn't consider in the code?
  \end{itemize}
  \medskip
  
  Poor scaling:
  \begin{itemize}
  \item a few unit tests per feature
  \item for $n$ features, $O(n)$ unit tests
  \item but testing \emph{interactions} between features requires $O(n^2),
    O(n^3), \ldots$ unit tests
  \end{itemize}
  \pause
  \bigskip
  

    \begin{minipage}{0.6\textwidth}
      Solution:
      \medskip
      
      \begin{quote}
        ``Don't write tests \\
        --- generate them!''
      \end{quote}
      John Hughes, co-author of the \emph{QuickCheck}
      PBT library
    \end{minipage}
    \begin{minipage}{0.3\textwidth}
      \hfill
      \includegraphics[width=0.8\textwidth]{images/john-hughes}
    \end{minipage}
  
\end{frame}

\begin{frame}[allowframebreaks]
  \frametitle{Property-based testing}

\begin{itemize}
\item Write \emph{properties} instead of specific tests
\begin{itemize}
\item should be universal, i.e.\@ hold for all values
\item should define the expected behaviour for \emph{all} cases
\end{itemize}
\item Specify \emph{generators} for the inputs
\item The testings framework runs the property with
  a large number of inputs
  \begin{itemize}
  \item testing fails if a \alert{counter-example} is found
  \item otherwise, testing succeeds
  \end{itemize}
\end{itemize}

\framebreak

\begin{itemize}
\item QuickCheck (2000): first PBT library (for Haskell)
\item Implementations for other languages
  \begin{description}
  \item[PropEr] for Erlang
  \item[ScalaCheck] for Scala
%  \item[QuickCheck-Core] for OCaml
  \item[Hypothesis] for Python
  \item[FsCheck] for F\#
  \item[JUnit-QuickCheck] for Java
  \item[RapidCheck] for C++
  \end{description}
\end{itemize}
\bigskip

\ldots many others: \url{https://en.wikipedia.org/wiki/QuickCheck}
\end{frame}


\begin{frame}[fragile]
  \frametitle{Example property}

  What can we say about the integer square root function?
  \pause
  \medskip

  \begin{block}{Property}
    Let $n$ be an arbitrary non-negative number;
    let $r = \texttt{isqrt}(n)$; then
    \[ r\geq0 \land r^2 \leq n \land (r+1)^2>n  \]
    i.e.\@ $r$ should be \emph{largest non-negative integer} such that
  $r^2 \leq n$.
  \end{block}
  \pause
  \medskip

  In Python:
\begin{semiverbatim}
from hypothesis import \ldots
import hypothesis.strategies as st
@given(\alert<5>{st.integers(min_value=0)})  \only<5->{\textsl{# non-negative integer}}
def test_isqrt(\alert<4>{n}):                \only<4->{\textsl{# for all n}}
    r = isqrt(n)
    assert \alert<6>{r>=0 and r**2<=n and (r+1)**2>n}   \only<6->{\textsl{# assertion}}
  \end{semiverbatim}

\end{frame}

\begin{frame}
  \frametitle{Properties in Hypothesis}

  \begin{itemize}
  \item Properties are \emph{functions}\ldots
  \item \ldots that should fail if the expected condition is not met
  \item Arguments are \emph{universally quantified}
  \item For each property:
    \begin{itemize}
  \item generate a large number of tests (100 by default)
  \item test data is randomly generated using \emph{strategies}
  (defined through \texttt{@given})
    \end{itemize}
  \item Module \texttt{hypothesis.strategies} provides:
    \begin{itemize}
    \item \emph{predefined strategies} for basic types
    \item methods for \emph{modifying} and \emph{combining} strategies
    \end{itemize}
  \end{itemize}
\bigskip

  \hfill (Cue demo.)
\end{frame}



\begin{frame}
  \frametitle{Strategies}

  \begin{description}
  \item[integers()] generate integers
  \item[booleans()] generate logical values 
  \item[text()] generate Unicode strings
  \item[lists($s$)] lists of elements given by strategy $s$
    % \item[tuples($s1,s2,\ldots$)] lists of elements given by strategies $s1,s2,\ldots$
    \item[\ldots] many others
  \end{description}
  \bigskip
  
  We can also:
  \begin{itemize}
  \item modify strategies using \emph{parameters} (e.g.\@ \texttt{min\_value})
  \item modify strategies by \emph{mapping} and \emph{filtering}
  \item combine them using some \emph{combinator functions}
  % \item sample them using the \texttt{.example()} method    
  \end{itemize} 
\end{frame}


\begin{frame}
  \frametitle{Another example}

  Let's test the interaction between \emph{list reverse}
  and \emph{append}.
  
  Consider $x, y$ two arbitrary lists:

  \[  reverse(x + y) = \alt<1>{\hspace{2ex}???\hspace{18ex}}{reverse(y) + reverse(x)} \]
  \pause

  Example:
  \[\begin{array}{ll}
      reverse([1,2] + [3,4]) &= reverse([3,4]) + reverse([1,2]) \\
                             &= [4,3] + [2,1]\\
                             &= [4,3,2,1]
  \end{array}\] 
\end{frame}

\begin{frame}[fragile]
  \frametitle{Testing with lists of integers}

\begin{verbatim}
intlist = st.lists(st.integers())
@given(intlist, intlist)
def test_reverse_append(x, y):
    assert reverse(x + y) == reverse(x) + reverse(y)
\end{verbatim}
\medskip
\pause
  
\begin{semiverbatim}
$ pytest basic.py -k test_reverse_append

==================== FAILURES ==========================
\alert{______________ test_reverse_append _____________________}

x = [0], y = [1]
\end{semiverbatim}
What happened\ldots ?  
\end{frame}

\begin{frame}[fragile]
  \frametitle{Checking expectations}


\begin{itemize}
\item We've written the property incorrectly!
\begin{semiverbatim}
  assert reverse(x + y) == reverse(\alert{x}) + reverse(\alert{y})
      \textsf{instead of}
  assert reverse(x + y) == reverse(y) + reverse(x)  
\end{semiverbatim}
\item Hypothesis found a counter-example for the wrong property:
   \[ reverse([0]+[1]) \neq reverse([0]) + reverse([1]) \]
\item This is the \emph{smallest} counter-example
  that falsifies the property
\item Hypothesis \emph{always} find this counter-example
  (regardless of random generation)!
\end{itemize}
\end{frame}

\begin{frame}[fragile,allowframebreaks]
  \frametitle{Shrinking}

Assume Hypothesis randomly generates
\begin{semiverbatim}
x = [2,2]
y = [1]
\end{semiverbatim}
This is a counter-example because:
\begin{semiverbatim}
      reverse([2,2]+[1]) \ensuremath{\neq}  reverse([2,2]) + reverse([1])
\end{semiverbatim}
\medskip

Hypothesis attempts to simplify
the counter-example (``shrinking'')
before presenting:
\begin{itemize}
\item removing elements from the lists
\item recursively shrinking the elements inside the lists
\end{itemize}
This is very useful to reduce ``noise'' from the randomly generated data.
\end{frame}

\begin{frame}
\frametitle{Shrinking in action}

\[ \xymatrix@-0.5pc{
 & \counter{[2,2],[1]}\ar[dl]\ar[d] \\
 \noncounter{[],[1]} &  \counter{[2],[1]}\ar[dl]\ar[dr] \\
 \noncounter{[],[1]} & &\alert<2->{\counter{[0],[1]}}\ar[dl]\ar[d]\ar[dr]
 &  \visible<2->{\alert{\text{minimal}}} \\
&  \noncounter{[],[1]} & \noncounter{[0],[]} & \noncounter{[0],[0]}
} 
\]

\begin{itemize}
\item Shrinking ends with  \texttt{x = [0]}, \texttt{y = [1]}
  (smaller lists are no longer counter-examples)
\item For this property this is \emph{minimal} counter-example
\item In general, shrinking only finds a \emph{local minimum}
\end{itemize}
\end{frame}


\begin{frame}[fragile]
\frametitle{A larger example}
\framesubtitle{Based on code from Ericsson}

SMS text packing:
\begin{itemize}
\item 7-bit characters;
\item transmitted using 8-bit \emph{bytes};
\item we can pack eight 7-bit caracteres into seven 8~bit bytes;
\item two functions:
\begin{verbatim}
pack(seq: bytes) -> bytes
unpack(seq: bytes) -> bytes
\end{verbatim} 
\end{itemize}

\begin{block}{``Round-trip'' property}
\begin{verbatim}
@given(...)
def test_round_trip(seq):
    assert unpack(pack(seq)) == seq
\end{verbatim}
\end{block}
\end{frame}

\begin{frame}[allowframebreaks]
  \frametitle{Conclusion}

\begin{itemize}
\item PBT philosophy: write \emph{properties} and \emph{generate} tests
\item Lightweight: implemented as libraries
\item Flexible: domain-specific languages (DSLs) for writing
    data generators and properties
 \item Scales to realistic software
    
\item Couples executable specifications with code
\item Useful for comunicating expectations among developers
\item Useful for finding subtle bugs in complex systems 
\end{itemize}
\end{frame}

\begin{frame}
  \frametitle{Challenges}
\begin{itemize}
\item Writting \emph{effective} properties and generators
  \begin{itemize}
  \item teaching developers pre- and post-conditions, invariants, etc.
  \item helping industry adopt a higher-skill technology 
  \item university CS/SE courses can play a significant rôle here
  \end{itemize}
\item PBT works best with well-structured software
\item Design systems around properties and not the other way around
\item Is validation of AI~generated code the ``killer app''
  for PBT? 
\end{itemize}
\end{frame}

\begin{frame}
  \frametitle{References}

  \begin{thebibliography}{9}
  \bibitem{QuickCheck00} Koen Claessen \& John Hughes: \newblock
    \emph{QuickCheck: A lightweight tool for random testing
      of Haskell programs}, ICFP 2000.
  \bibitem{telecoms} Thomas Arts, John Hughes \& Joakim Johansson: \newblock
    \emph{Testing Telecoms Software with Quviq QuickCheck}, Erlang'06.
\end{thebibliography}
\end{frame}

\begin{frame}
\begin{center}
  \Huge Extra slides
\end{center} 
\end{frame}



\begin{frame}[fragile]
  \frametitle{Generating data}

\begin{verbatim}
>>> integers().example()
848041

>>> lists(integers(min_value=0, max_value=100)).example()
[2, 29, 54, 66, 1, 27, 77, 81, 51, 18, 18]

>>> lists(integers().map(lambda x:x*2)).example()
[6668, -38, 1081651134, -6590]

>>> lists(integers()).map(sorted).example()
[-6913, -59, 37, 77, 90, 25088]

>>> lists(booleans()).filter(lambda lst:lst!=[]).example()
[False, False, True, False]
\end{verbatim}
  
\end{frame}


\begin{frame}
  \frametitle{Stateful programs}

We can also use PBT to test programs that:
\begin{enumerate}
\item modify state;
\item read and write files;
\item use network services, databases, etc.
\end{enumerate}
\end{frame}

\begin{frame}
  \frametitle{Testing stateful programs}

\begin{itemize}
\item Generate sequences of commands
\item Specify behaviour using a functional model (state machine)
\item Compare the execution against the model
\end{itemize}

\begin{center}
\includegraphics[width=0.75\textwidth]{images/imperative.png}
\end{center}
\end{frame}



\begin{frame}
  \frametitle{Industrial use example}

\begin{itemize}
\item Ericsson Media proxy
\item Establish telephony connection throught a firewall
\item Tested with Erlang QuickCheck (Quviq.com)
\item Adding and removing paricipants in a call
\item Random counterexample with 160 commands
\item Shrunk automatically to 7 commands
\end{itemize}

\begin{center}
\includegraphics[width=\textwidth]{images/media1}
\end{center}
\end{frame}








\end{document}

